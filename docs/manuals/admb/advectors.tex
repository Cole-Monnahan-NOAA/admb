% $Id: advectors.tex 204 2009-04-20 01:59:36Z jsibert $
%
% Author: David Fournier
% Copyright (c) 2008 Regents of the University of California
%

\X{\fontindexentry{tt}{init\_number\_vector}}
\X{\fontindexentry{tt}{init\_bounded\_number\_vector}}
\X{\fontindexentry{tt}{init\_vector\_vector}}
\X{\fontindexentry{tt}{init\_bounded\_vector\_vector}}
\X{\fontindexentry{tt}{init\_matrix\_vector}}
\X{\fontindexentry{tt}{init\_bounded\_matrix\_vector}}
This chapter introduces three new \ADM\ types. They are
\begin{lstlisting}
init_number_vector
init_vector_vector 
init_matrix_vector
\end{lstlisting}
plus the bounded versions of these
\begin{lstlisting}
init_bounded_number_vector
init_bounded_vector_vector 
init_bounded_matrix_vector
\end{lstlisting}
To understand the usefulness of these objects, consider an application
that has two \texttt{init\_number} objects:
\begin{lstlisting}
PARAMETER_SECTION
  init_bounded_number a1(0.2,1.0,1)
  init_bounded_number a2(-1.0,0.3,2)
\end{lstlisting} 
This creates two bounded numbers with different
upper and lower bounds becoming active in different phases
of the minimization.
Now, however, suppose that the number of numbers we wish to
have in the model depends on some integer read in at run time, such as:
\begin{lstlisting}
DATA_SECTION
  init_int n
 //  ...

PARAMETER_SECTION
  // want to have n numbers
  init_bounded_number a1(0.2,1.0,1)
  init_bounded_number a2(-1.0,0.3,2)
  // ....
  init_bounded_number an(-4.0,-3.0,n)
\end{lstlisting} 
The above code is a sketch of what we want to 
achieve. It cannot be accomplished with that kind of coding,
of course, because at compile time, we don't have the value for $n$,
and in any event, if $n$ is large, this sort of coding is boring.
Dynamic arrays are the answer to this problem. One could try
the following:
\begin{lstlisting}
DATA_SECTION
  init_int n
 //  ...

PARAMETER_SECTION
  // want to have n numbers
  init_bounded_vector a(1,n,-1.0,1.0,1)
\end{lstlisting} 
but this won't work, because for an \texttt{init\_bounded\_vector},
the bounds and the starting phase are the same for all components of
the vector. The \texttt{init\_bounded\_number\_vector} class
is intended to solve this problem.
\begin{lstlisting}
DATA_SECTION
  init_int n
 //  ...

PARAMETER_SECTION
  // need to create some vectors to hold the bounds and
  // phase numbers
 LOC_CALCS
  dvector lb(1,n); 
  dvector ub(1,n); 
  ivector ph(1,n); 
  // get the desired values into lb,ub,ph somehow 
  lb.fill_seqadd(1,0.5);
  ub.fill_seqadd(2,0.5);
  ph.fill_seqadd(1,1);
 END_CALCS
  init_bounded_number_vector a(1,n,lb,ub,ph)
\end{lstlisting} 
Then \texttt{a(1)} is an object of type
\texttt{init\_bounded\_number} with bounds \texttt{lb(1)} and 
\texttt{ub(1)} becoming active in phase \texttt{ph(1)}. 
Any of these three fields can be replaced with a number or
integer if the bound or phase number is constant, such as
\begin{lstlisting}
  init_bounded_number_vector a(1,n,1.0,ub,2)
\end{lstlisting} 
where the lower bound is \texttt{1.0} and the
phase number is~\texttt{2}.


